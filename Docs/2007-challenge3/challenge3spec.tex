\documentclass[11pt]{article}
\usepackage{geometry}
\geometry{letterpaper,vmargin=1.0 in,hmargin=1.0 in}
\usepackage{graphicx}
\usepackage{amssymb}
\usepackage{amsbsy}
\usepackage{epstopdf}
\usepackage{url}

\DeclareGraphicsRule{.tif}{png}{.png}{`convert #1 `dirname #1`/`basename #1 .tif`.png}

\title{Challenge 3 Specification}
\author{MLDC taskforce}
\date{Dec 6, 2007}

\begin{document}
\maketitle

\subsection*{3.1: Galaxy with chirping binaries}

A two-year dataset ($2^{22}$ samples with $15$ s sampling time) with signals from $\sim 6 \times 10^7$ Galactic binaries.

The binaries are drawn from a randomized Nelemans {\it et al.} population of 26 million detached systems and 34 million
interacting system~\cite{nyz01,nyz04}. The orbital frequency may be increasing or decreasing depending on whether gravitational
radiation or mass transfer is the dominant mechanism. The evolution of the frequency is treated as being linear (i.e., $\dot{f}$
is a relevant parameter, but $\ddot{f}$ is not). The set includes also 20 ``verification binaries''
chosen from the original Nelemans population, with randomized extrinsic parameters and randomly tweaked intrinsic
parameters (so they will not have exactly the same frequency or sky location as sources in the general population).
The parameters for these pseduo verification binaries are given in Table 1.

\begin{table}
\caption{Fixed parameters for the 20 pseudo verification binaries}
\begin{center}
\begin{tabular}{|r@{\hspace*{0.1in}}|r@{\hspace*{0.1in}}|r@{\hspace*{0.1in}}|r@{\hspace*{0.1in}}|r@{\hspace*{0.1in}}|}
\hline
\hspace*{0.1in} $f \quad \quad$ & $\dot f \quad$ & $\beta \quad$ & $\lambda \quad$ & $A \quad$ \\
\hline
0.0043417274 &	7.3561e-16  &	0.68207	  &	0.53347	  &	5.9350e-22  \\
0.0037343941 &	-2.0297e-17 &	0.10474	  &	0.62924	  &	2.6135e-22  \\
0.0131529582 &	8.8204e-15  &	-0.98805  &	3.9268	  &	1.5156e-22  \\
0.0039699876 &	-3.3365e-17 &	-0.21254  &	4.7720	  &	9.9415e-23   \\
0.0045022047 &	1.3540e-15  &	0.031996  &	4.7446	  &	8.9184e-23   \\
0.0026610293 &	3.2306e-17  &	1.1472	  &	0.0095050  &	9.7080e-23   \\
0.0028695591 &	-4.3640e-18 &	-0.92861  &	3.8856	  &	1.3958e-23   \\
0.0062057876 &	-2.8732e-16 &	1.0398	  &	6.2805	  &	5.6116e-23   \\
0.0052955094 &	4.0514e-16  &	1.1159	  &	6.0834	  &	1.0108e-22   \\
0.0004639037 &	5.8521e-20  &	0.68715	  &	5.8248	  &	4.6955e-22   \\
0.0097827541 &	4.2655e-15  &	0.76860	  &	5.3829	  &	4.9380e-23    \\
0.0031386229 &	4.4597e-17  &	-0.58442  &	4.3714	  &	1.8405e-23    \\
0.0090517193 &	-3.1886e-15  &	-0.85312  &	3.9630	  &	3.3825e-23    \\
0.0030861822 &	3.1552e-17  &	-0.72290  &	2.3212	  &	4.2400e-23    \\
0.0022225674 &	2.5908e-17  &	-1.0163	  &	2.1245	  &	5.1583e-23    \\
0.0033341192 &	-9.9299e-18 &	1.4141	  &	2.2510	  &	1.2462e-22    \\
0.0023149283 &	-1.1627e-17 &	-0.56303  &	1.9615	  &	8.3458e-23    \\
0.0024189047 &	1.1142e-17  &	-0.14195  &	1.6711	  &	2.7004e-23    \\
0.0005542018 &	5.2244e-20  &	0.24866	  &	2.3023	  &	3.4189e-22    \\
0.0013821261 &	3.3793e-18  &	1.3093	  &	5.0762	  &	2.8972e-22    \\
\hline
\end{tabular}
\end{center}
\end{table}

The instrument noise is secondary-only ({\it i.e.} Laser phase noise is assumed to have been cancelled by the TDI
construction), Gaussian, stationary, and equal on all spacecraft. 

\subsection*{3.2: Massive--Black-Hole binaries over Galactic confusion}

A two-year dataset ($2^{22}$ samples with $15$ s sampling time) with signals from U[4,6] spinning MBH binary inspirals.

The spinning-binary signals are modeled as 2PN circular adiabatic inspirals with uncoupled orbital frequency evolution
and spin and orbital precession. No higher-PN harmonics are included. [\textbf{Neil and Stas: add more detail, link to full descriptive document}] The end of the inspiral is handled with an exponential taper, as in Challenge 2. Masses, SNRs, and times of coalescence
are chosen as in Challenge 2; spin amplitudes are drawn from U[0,1], and spin angles are randomized over spheres.
Sky position, polarization, extrinsic parameters are random over spheres.

The dataset includes also a Galactic confusion background generated from the same population as Challenge 3.1, but withholding
all binaries with individual SNR $> 5$ relative to the combined instrument plus galactic confusion
noise. The confusion noise estimate was derived using the BIC analysis code used in the evaluation of Challenge 2.
Only detached systems are used to generate the confusion background as the confusion estimate from Challenge 2 did
not include interacting systems. In any event, we do no expect interacting systems to be a significant contribution
to the confusion noise since they typically have very small chirp masses, and hence amplitudes, compared to the detached systems.

Instrument noise is secondary-only, Gaussian, stationary, and equal on all spacecraft. 

\subsection*{3.3: EMRIs}

A single two-year datasets ($2^{22}$ samples with $15$ s sampling time) containing the signals from 5 ``MLDC EMRIs'', with parameters drawn
as in Challenge 1B: compact object mass $m$ in U$[9.5, 10.5] \times M_{\odot}$, spin of central BH $S/M$ in U[0.5, 0.7], time at plunge
in U$[2^{21}, 2^{22}] \times 15$ sec, and eccentricity at plunge in U[0.15, 0.25]. The central black holes masses are chosen so that
one system has $M$ in U$[0.95, 1.05] \times 10^7 M_{\odot}$, two systems have $M$ in U$[4.75, 5.25] \times 10^6 M_{\odot}$ and
two systems have $M$ in U$[0.95, 1.05] \times 10^6 M_{\odot}$. In addition to having mutliple EMRIs in a single data set, the
other new wrinkle for this challenge is that the SNRs are much lower: SNR in U[10,50].
The number of eccentric-orbit harmonics does not vary with eccentricity
along the evolution (as was the case in Challenge 2 and 1B), but is fixed at 5. Sky position, polarization,
extrinsic parameters are random over spheres.

Instrument noise is secondary-only, Gaussian, stationary, and equal on all spacecraft. 

\subsection*{3.4: Bursts}

One ``MLDC month''-long dataset ($2^{21}$ samples with 1 s sampling time) with Poisson-distributed cosmic-string--cusp bursts, defined as in the accompanying document [\textbf{To be included as a chapter here}]. The Poisson even rate is 5 per ``MLDC month''. SNRs will be uniformly distributed between 10 and 100. The logarithm of the maximum frequency (see accompanying document) will be uniformly distributed as $\log_{10} f_\mathrm{max} \in [-3,1.0]$. Sky position and polarization are random over spheres.

Instrument noise is Gaussian and stationary; it includes secondary noise, where the levels of proof-mass and photo-detector noises are randomized independently on each optical bench by a uniform draw in $[-20,+20]\%$ w.r.t.\ their nominal value; it also includes laser noise, also randomized, with nominal amplitude reduced to $10\times$ the nominal amplitude of secondary noises at 1 mHz [\textbf{Michele: check}.] The datasets include the standard $X$, $Y$, $Z$ TDI observables \emph{and} the 12 inter-spacecraft and inter-spacecraft raw phase measurements. The datasets will be distributed only as fractional-frequency--fluctuation time series (i.e., as \emph{Synthetic LISA} datasets).

\subsection*{3.5: Stochastic Backgrounds}

One ``MLDC month''-long dataset ($2^{21}$ samples with 1 s sampling time [\textbf{Michele: check}]) with an isotropic stochastic background with $S_h(f) \propto f^{-3}$, realized by placing 192 ``pseudosources'' (independent pseudorandom processes for both $h_+$ and $h_\times$, with $f^{-3}$ spectra) at Healpix-distributed points across the sky [\textbf{Michele: ref.}]. The level of the background is raised to 3--6 times the secondary noise at the best frequency [\textbf{Michele: check}]

The instrument noise and the distributed observables are handled as in Challenge 3.4. The datasets will be released in versions produced by \emph{Synthetic LISA} and by the APC's \emph{LISACode}.

\begin{thebibliography}{99}
\bibitem{nyz01} G. Nelemans, L. R. Yungelson \& S. F. Portegies Zwart, A\&A {\bf 375}, 890 (2001).
\bibitem{nyz04} G. Nelemans, L. R. Yungelson \& S. F. Portegies Zwart, Mon. Not. Roy. Astron. Soc.
{\bf 349} 181, (2004).
\end{thebibliography}

\end{document}  