\documentclass[12pt]{article}
\usepackage{times,latexsym}
\usepackage{amsmath}
\usepackage{stmaryrd}
\usepackage{graphicx}
\usepackage{epsfig}

\pagestyle{empty}
\setlength{\parindent}{0.5cm}
\setlength{\textwidth}{6.5in}
\setlength{\textheight}{8.6in}
\setlength{\oddsidemargin}{0in}
\setlength{\leftmargin}{0.1in}
\setlength{\parskip}{0cm}
\setlength{\evensidemargin}{0in}
\setlength{\topmargin}{0.1in}

\begin{document} 

\section{Challenge 3.1: Galactic Foreground}

The Challenge 3.1 galactic foreground data sets are direct descendants of Challenge 2.1.
The only modifications are that the underlying population synthesis model now includes interacting
AM CVn systems in addition to the detached systems used in the earlier
challenges, and the leading term in the frequency evolution is now
included in the signal model.

The starting point for the simulations are two large data files,
provided by Gijs Nelemans, that contain a list of detached and
interacting systems produced by the population synthesis codes
described in Refs.~\cite{Nelemans:2001hp, Nelemans:2003ha}. The
columns in these data files list the component masses $M,m$ of
the heavier and lighter component, the initial orbital
period $P_0$ and its first time derivative $\dot{P}_0$, the
time rate of change of the donor mass $\dot{m}$, the galactic latitude
and longitude, and the distance $d$ to the system. There are a total
of  26.1 million detached systems and 34.2 million interacting systems
in the input data files. Recent work by Roelofs, Nelemans and
Groot~\cite{Roelofs:2007rn} suggests that the population
synthesis model of Ref.~\cite{Nelemans:2003ha} over predicts the
number of AM CVn systems by a factor of 5 or 10, but this
correction has not been applied to the Challenge 3.1 simulations.

Some of the information in the population files is redundant since the
rate of orbital evolution can be computed from the masses, orbital
period and mass transfer rate by
\begin{equation}
\frac{\dot{f}}{f} = -\frac{\dot{P}}{P} = 
\frac{96}{3}(\pi f)^{8/3}{\cal M}^{5/3}  
- 3 \left(1 - \frac{m}{M}\right) \frac{\dot{m}}{m}
\end{equation}
where $f=2/P$ is the gravitational wave frequency and
${\cal M} = (mM)^{3/5}/(m+M)^{1/5}$ is the chirp mass. For the detached
systems $\dot{m}=0$ and the orbits decay through
gravitational wave emission $(\dot{f} >0)$, while the interacting
systems are typically in a state of equilibrium mass transfer such that
the orbits expand $(\dot{f} < 0)$.

\begin{center}
\begin{table}\label{galactic}
\begin{tabular}{l|c}
\hline
\hline
\multicolumn{2}{c}{{\bf Galactic binary source descriptors}} \\
\hline
\hline
$f_0$    & initial GW frequency\\ 
$\dot{f}_0$ & initial time-derivative of GW frequency\\
$\beta$ & ecliptic latitude \\
$\lambda$ & ecliptic longitude\\
${\cal A}_0$ & initial signal amplitude\\
$\iota$ & inclination angle\\
$\psi$ & polarization angle\\
$\varphi_0$ &  initial GW phase\\
\hline \hline
\end{tabular} \\
\end{table}
\end{center}

The population files are read in and converted into the MLDC galactic
binary source descriptors listed in Table~\ref{galactic}. Quantities
such as the inclination angle, polarization angle and orbital phase
are drawn at random. Galactic coordinates are converted to Ecliptic
coordinates and given a small random tweak. Similarly, the orbital period
and period derivative are converted to initial gravitational wave frequency
and frequency derivative with a small random tweak. These random
perturbations are large enough to render the population files useless
as answer keys, but small enough so as not to upset the overall
parameter distributions. The initial amplitude is computed using
the relation
\begin{equation}\label{amp}
{\cal A}_0 = \frac{2G^{5/3}}{c^4 d}(2\pi f_0)^{2/3} {\cal M}^{5/3} \, .
\end{equation}
The barycenter waveform in the MLDC polarization basis can then be written:
\begin{eqnarray}\label{bary_galactic}
h_+(t) &=& {\cal A}_0( \cos(2\psi)(1 + \cos^2{\iota})\cos\Psi(t) - 2\sin(2 \psi)\cos\iota \sin\Psi(t) ), \\
h_\times(t) &=& -{\cal A}_0(2 \cos (2\psi)\cos\iota \sin\Psi(t)  - \sin (2 \psi)(1 + \cos^2{\iota})\cos\Psi(t)),
\end{eqnarray}
where
\begin{equation}
\Psi(t) = 2\pi f_0 t + 2\pi \dot{f}_0 t^2 + \varphi_0 \, .
\end{equation}

Processing the barycenter waveform (\ref{bary_galactic}) for each source through the instrument simulators
would take many processor years to complete, so an alternative method~\cite{Cornish:2007if} had to be devised to produce the
galactic data sets. On solution is to use re-write the LISA phase meter readouts in a fast-slow decomposition of
the form
\begin{equation}
y_{ij}(t) = C(t) \cos(2 \pi f_0 t) + S(t) \sin(2 \pi f_0 t) \, .
\end{equation}
The functions $C(t)$ and $S(t)$ describe slowly varying effects such as the time variation of the
antenna patterns, Doppler shifting of the barycenter frequency and the intrinsic frequency evolution.
These slow varying terms can be sampled very sparsely and numerically Fourier transformed. The fast
varying sine and cosine terms can be analytically Fourier transformed. The results are then
convolved to produce the full phase meter output, which can then be used to construct the desired
TDI variables. This algorithm is three to four orders of magnitude faster than the general
purpose instrument simulators. Full details can be found in Ref.~\cite{Cornish:2007if}, and the
source code is part of {\it lisatools} repository.




\begin{thebibliography}{99}

%\cite{Nelemans:2001hp}
\bibitem{Nelemans:2001hp}
  G.~Nelemans, L.~R.~Yungelson and S.~F.~Portegies Zwart,
  %``The gravitational wave signal from the galactic disk population of
  %binaries containing two compact objects,''
  arXiv:astro-ph/0105221.

%\cite{Nelemans:2003ha}
\bibitem{Nelemans:2003ha}
  G.~Nelemans, L.~R.~Yungelson and S.~F.~Portegies Zwart,
  %``Short-period AM CVn systems as optical, X-ray and gravitational wave
  %sources,''
  Mon.\ Not.\ Roy.\ Astron.\ Soc.\  {\bf 349}, 181 (2004)
  [arXiv:astro-ph/0312193].

%\cite{Roelofs:2007rn}
\bibitem{Roelofs:2007rn}
  G.~H.~A.~Roelofs, G.~Nelemans and P.~J.~Groot,
  %``The population of AM CVn stars from the Sloan Digital Sky Survey,''
  arXiv:0709.2951 [astro-ph].

%\cite{Cornish:2007if}
\bibitem{Cornish:2007if}
  N.~J.~Cornish and T.~B.~Littenberg,
  %``Tests of Bayesian Model Selection Techniques for Gravitational Wave
  %Astronomy,''
  Phys.\ Rev.\  D {\bf 76}, 083006 (2007)
  [arXiv:0704.1808 [gr-qc]].
  %%CITATION = PHRVA,D76,083006;%%

\end{thebibliography}

\end{document}
